\documentclass{article}
\usepackage{array}
\usepackage{tabularx}
\usepackage{lipsum}
%\everymath{\displaystyle}
\usepackage{xparse}
\NewDocumentEnvironment{halfequ}{
  +b
}{
  \begin{minipage}{0.5\linewidth}
  \begin{equation}
  #1
}{
  \end{equation}
  \end{minipage}
}
\begin{document}
\thispagestyle{empty}
a$hh$b
\begin{table}[ht]
	\begin{tabular}{|p{10em}|l|}\hline
		This is a test, this is a test, this is a test, this is a test, this is a test, this is a test, this is a test & b\\ \hline
	\end{tabular}
	\newline
	\begin{tabularx}{\linewidth}{|X|l|}\hline
		This is a test, this is a test, this is a test, this is a test, this is a test, this is a test, this is a test & b\\ \hline
	\end{tabularx}
\end{table}
\section{hello}\label{sec:test}
This is a test\ref{sec:test}.
\newpage
\section{hello}
\romannumeral255
\csname\string\TeX\endcsname \csname TeX\endcsname '''a'\thinspace'' '''a'\,'' {`}``a`\,``, \lq\lq\ \lq a\rq\ \rq\rq\ a
'''b'$\,$'' '''b'\,'' a a b c

\begin{halfequ}
f(x)=x_{2}
\end{halfequ}
\begin{halfequ}
a=b
\end{halfequ}

\'{}\TeX\'{}

\begin{equation}
a=\bigl[\frac{\frac{1}{n}}{\sqrt{n}}\bigr]
\end{equation}
\begin{equation}
	a=\Bigl[\frac{\frac{1}{n}}{\sqrt{n}}\Bigr]
\end{equation}
\begin{equation}
	a=\biggl[\frac{\frac{1}{n}}{\sqrt{n}}\biggr]
\end{equation}
\begin{equation}
	a=\Biggl[\frac{\frac{1}{n}}{\sqrt{n}}\Biggr]
\end{equation}
\begin{equation}
afa
\end{equation}

\begin{equation}
a
\end{equation}
a\~{\space}bg a\verb*|~|b a\texttildelow b a\textasciitilde b

\end{document}
%%% Local Variables:
%%% mode: latex
%%% TeX-master: t
%%% End:

