%%=============setting,设置自己的队号和选题============
\gdef\MCMcontrol{3256489}%队号
\newcommand{\problem}{A}%选题

\newcommand{\control}{\MCMcontrol}
\newcommand{\team}{Team \#\ \MCMcontrol}
\newcommand{\headset}{{\the\year}\\MCM/ICM\\Summary Sheet}

%==========定义摘要,摘要的标题可自定义===================
\renewenvironment{abstract}[1]{%
	\small
	\begin{center}%
		{\large\bfseries #1\vspace{-.5em}}%
\end{center}}
{}
\newcommand\keywords[1]{%
	\begingroup
	\par
	\noindent\textbf{Keywords:} #1\par
	\endgroup
}

% 目录居中的重定义
\makeatletter
\renewcommand\tableofcontents{%
	\centerline{\normalfont\Large\bfseries\contentsname%
		\@mkboth{%
			\MakeUppercase\contentsname}{\MakeUppercase\contentsname}}%
	\vskip 3ex%
	\@starttoc{toc}%
	\thispagestyle{fancy}
	\clearpage}
\makeatother

\usepackage[toc, page, title, titletoc, header]{appendix}
\usepackage{graphicx}
\graphicspath{{figures/}{img/}}

\usepackage{amsmath,amssymb,amsfonts,amsthm}
\newtheorem{Theorem}{Theorem}[section]
\newtheorem{Lemma}[Theorem]{Lemma}
\newtheorem{Corollary}[Theorem]{Corollary}
\newtheorem{Proposition}[Theorem]{Proposition}
\newtheorem{Definition}[Theorem]{Definition}
\newtheorem{Example}[Theorem]{Example}


%==========设置代码格式===================
\usepackage{xcolor}
\usepackage{listings}
\definecolor{grey}{rgb}{0.8,0.8,0.8}
\definecolor{darkgreen}{rgb}{0,0.3,0}
\definecolor{darkblue}{rgb}{0,0,0.3}
\def\lstbasicfont{\fontfamily{pcr}\selectfont\footnotesize}
\lstset{%
	% numbers=left,
	% numberstyle=\small,%
	showstringspaces=false,
	showspaces=false,%
	tabsize=4,%
	frame=lines,%
	basicstyle={\footnotesize\lstbasicfont},%
	keywordstyle=\color{darkblue}\bfseries,%
	identifierstyle=,%
	commentstyle=\color{darkgreen},%\itshape,%
	stringstyle=\color{black}%
}
\lstloadlanguages{C,C++,Java,Matlab,Mathematica}

\usepackage{geometry}
\geometry{a4paper, margin = 1.2in}

%==========设置页眉格式===================
\usepackage{fancyhdr,lastpage}
\pagestyle{fancy}
\fancyhf{}
\lhead{\small\sffamily \team}
\rhead{\small\sffamily Page \thepage\ of \pageref{LastPage}}
\setlength\parskip{.5\baselineskip}

\usepackage{hyperref}
\usepackage{mathptmx}% newtxtext
\usepackage{lipsum}
\title{The \LaTeX{} Template for MCM Version 1}
\author{\small \href{http://www.latexstudio.net/}
	{\includegraphics[width=7cm]{mcmthesis-logo}}}
\date{\today}