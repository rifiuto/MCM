
%% -----------------------------------
%%
%%
%% Copyright (C)
%%     2022     by latexstudio.net
%%
%%
\documentclass[12pt]{article}
%% 导言区
%%=============setting,设置自己的队号和选题============
\gdef\MCMcontrol{3256489}%队号
\newcommand{\problem}{A}%选题

\newcommand{\control}{\MCMcontrol}
\newcommand{\team}{Team \#\ \MCMcontrol}
\newcommand{\headset}{{\the\year}\\MCM/ICM\\Summary Sheet}

%==========定义摘要,摘要的标题可自定义===================
\renewenvironment{abstract}[1]{%
	\small
	\begin{center}%
		{\large\bfseries #1\vspace{-.5em}}%
\end{center}}
{}
\newcommand\keywords[1]{%
	\begingroup
	\par
	\noindent\textbf{Keywords:} #1\par
	\endgroup
}

% 目录居中的重定义
\makeatletter
\renewcommand\tableofcontents{%
	\centerline{\normalfont\Large\bfseries\contentsname%
		\@mkboth{%
			\MakeUppercase\contentsname}{\MakeUppercase\contentsname}}%
	\vskip 3ex%
	\@starttoc{toc}%
	\thispagestyle{fancy}
	\clearpage}
\makeatother

\usepackage[toc, page, title, titletoc, header]{appendix}
\usepackage{graphicx}
\graphicspath{{figures/}{img/}}

\usepackage{amsmath,amssymb,amsfonts,amsthm}
\newtheorem{Theorem}{Theorem}[section]
\newtheorem{Lemma}[Theorem]{Lemma}
\newtheorem{Corollary}[Theorem]{Corollary}
\newtheorem{Proposition}[Theorem]{Proposition}
\newtheorem{Definition}[Theorem]{Definition}
\newtheorem{Example}[Theorem]{Example}


%==========设置代码格式===================
\usepackage{xcolor}
\usepackage{listings}
\definecolor{grey}{rgb}{0.8,0.8,0.8}
\definecolor{darkgreen}{rgb}{0,0.3,0}
\definecolor{darkblue}{rgb}{0,0,0.3}
\def\lstbasicfont{\fontfamily{pcr}\selectfont\footnotesize}
\lstset{%
	% numbers=left,
	% numberstyle=\small,%
	showstringspaces=false,
	showspaces=false,%
	tabsize=4,%
	frame=lines,%
	basicstyle={\footnotesize\lstbasicfont},%
	keywordstyle=\color{darkblue}\bfseries,%
	identifierstyle=,%
	commentstyle=\color{darkgreen},%\itshape,%
	stringstyle=\color{black}%
}
\lstloadlanguages{C,C++,Java,Matlab,Mathematica}

\usepackage{geometry}
\geometry{a4paper, margin = 1.2in}

%==========设置页眉格式===================
\usepackage{fancyhdr,lastpage}
\pagestyle{fancy}
\fancyhf{}
\lhead{\small\sffamily \team}
\rhead{\small\sffamily Page \thepage\ of \pageref{LastPage}}
\setlength\parskip{.5\baselineskip}

\usepackage{hyperref}
\usepackage{mathptmx}% newtxtext
\usepackage{lipsum}
\title{The \LaTeX{} Template for MCM Version 1}
\author{\small \href{http://www.latexstudio.net/}
	{\includegraphics[width=7cm]{mcmthesis-logo}}}
\date{\today}
%=========设置主要、次要文件===============
\usepackage{subfiles}
\usepackage{enumitem}
\setlength{\parskip}{0.5em}
\usepackage{tabularray}
% bind figure to section
\counterwithin{figure}{section}
\counterwithin{equation}{subsection}
\begin{document}
	
	%% 摘要环境
	\subfile{./segments/abstract_}
	\newpage
	\tableofcontents
	
	
	%==========设置正文格式===================
	
	\subfile{./segments/introduction_} %
	
	% 分析
	\subfile{./segments/analysis_}
	
	% 计算
	\subfile{./segments/calculate_}
	
	% 模型结果
	\subfile{./segments/model_result.tex} %
	
	%
	\subfile{./segments/validate_model}
	
	% 总结
	\subfile{./segments/conclusion_}
	
	
	% \section{A Summary}
	% \lipsum[6]
	
	% % 模型评估
	% \subfile{./segments/evaluate_mode}
	
	
	% % 模型的优点和缺点
	% \subfile{./segments/strengths_weaknesses}
	
	
	
	
	\begin{thebibliography}{99}
		\bibitem{1} Everett M.Rogers. Diffusion of Innovations. New York:The Free Press, 1983.
		\bibitem{2} Duan Peng. Theory and Application of Discrete Choice Model [D]. Nankai University, 2010.
		Addison-Wesley Publishing Company, 1986.
		\bibitem{3} Feng Jiandong, Wang Hao. Investigation and Analysis on the Characteristics and Travel Willingness of Bus Passengers in Zhengzhou City [J]. Transportation Technology and Economy, 2015, 17(04): 64-70.
	\end{thebibliography}
	
	
	
	% \begin{appendices}
	% 	\section{First appendix}
	% 	\lipsum[13]
	% 	Here are simulation programmes we used in our model as follow.\\
	% 	\textbf{\textcolor[rgb]{0.98,0.00,0.00}{Input matlab source:}}
	% 	\lstinputlisting[language=Matlab]{./code/mcmthesis-matlab1.m}
	% 	\section{Second appendix}
	% 	some more text \textcolor[rgb]{0.98,0.00,0.00}{\textbf{Input C++ source:}}
	% 	\lstinputlisting[language=C++]{./code/mcmthesis-sudoku.cpp}
	% \end{appendices}
	
	
	
	
\end{document}

%%% Local Variables:
%%% mode: latex
%%% TeX-master: t
%%% End:
