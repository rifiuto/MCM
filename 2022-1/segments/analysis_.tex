\documentclass[../mcmpaper]{subfiles}
\begin{document}
	\section{Reasonable Assumptions}
	\subsection*{\itshape Assumptions about the data provided.}
    \begin{enumerate}[left=0pt .. \parindent]
        \item The uploaded location (latitude and longitude) has high accuracy and no deviation from the map on Google\footnote{\url{https://www.google.com}}
        \item The judgments made by the WSDA based on photos and the corresponding comments is accurate, and there is no error in recognition;
        \item The date of the photo was taken is always equal to the corresponding detection date, i.e., there is no delay in uploading photos;
        \item Whether a photo contains an Asian giant hornet, is independent of the status of other photos.
    \end{enumerate}
    \subsection*{Assumptions about the behavior of Asian giant hornet.}
    \begin{enumerate}[left=0pt .. \parindent]
        \item When the surrounding geographical environment is similar, the Asian giant hornet preys randomly, as well as the population dispersal. At the same time, the species tends to prey and migrate nearby.
        \item The Asian giant hornet cannot build nests or prey in the water;
        \item The environment is similar in different regions, so fitted hyperparameters in a certain small area can be regarded as constants. Therefore, hyperparameters can be promoted throughout Washington State.
    \end{enumerate}
\end{document}
%%% Local Variables:
%%% mode: latex
%%% TeX-master: "../mcmpaper"
%%% End:
