%% -----------------------------------
%%
%%
%% Copyright (C)
%%     2022     by latexstudio.net
%%
%%
\documentclass[12pt]{article}
%% preamble
\makeatletter
\def\input@path{{/home/sayno/mcm/2022-4/}}
\makeatother
%%=============setting,设置自己的队号和选题============
\gdef\MCMcontrol{3256489}%队号
\newcommand{\problem}{A}%选题

\newcommand{\control}{\MCMcontrol}
\newcommand{\team}{Team \#\ \MCMcontrol}
\newcommand{\headset}{{\the\year}\\MCM/ICM\\Summary Sheet}

%==========定义摘要,摘要的标题可自定义===================
\renewenvironment{abstract}[1]{%
	\small
	\begin{center}%
		{\large\bfseries #1\vspace{-.5em}}%
\end{center}}
{}
\newcommand\keywords[1]{%
	\begingroup
	\par
	\noindent\textbf{Keywords:} #1\par
	\endgroup
}

% 目录居中的重定义
\makeatletter
\renewcommand\tableofcontents{%
	\centerline{\normalfont\Large\bfseries\contentsname%
		\@mkboth{%
			\MakeUppercase\contentsname}{\MakeUppercase\contentsname}}%
	\vskip 3ex%
	\@starttoc{toc}%
	\thispagestyle{fancy}
	\clearpage}
\makeatother

\usepackage[toc, page, title, titletoc, header]{appendix}
\usepackage{graphicx}
\graphicspath{{figures/}{img/}}

\usepackage{amsmath,amssymb,amsfonts,amsthm}
\newtheorem{Theorem}{Theorem}[section]
\newtheorem{Lemma}[Theorem]{Lemma}
\newtheorem{Corollary}[Theorem]{Corollary}
\newtheorem{Proposition}[Theorem]{Proposition}
\newtheorem{Definition}[Theorem]{Definition}
\newtheorem{Example}[Theorem]{Example}


%==========设置代码格式===================
\usepackage{xcolor}
\usepackage{listings}
\definecolor{grey}{rgb}{0.8,0.8,0.8}
\definecolor{darkgreen}{rgb}{0,0.3,0}
\definecolor{darkblue}{rgb}{0,0,0.3}
\def\lstbasicfont{\fontfamily{pcr}\selectfont\footnotesize}
\lstset{%
	% numbers=left,
	% numberstyle=\small,%
	showstringspaces=false,
	showspaces=false,%
	tabsize=4,%
	frame=lines,%
	basicstyle={\footnotesize\lstbasicfont},%
	keywordstyle=\color{darkblue}\bfseries,%
	identifierstyle=,%
	commentstyle=\color{darkgreen},%\itshape,%
	stringstyle=\color{black}%
}
\lstloadlanguages{C,C++,Java,Matlab,Mathematica}

\usepackage{geometry}
\geometry{a4paper, margin = 1.2in}

%==========设置页眉格式===================
\usepackage{fancyhdr,lastpage}
\pagestyle{fancy}
\fancyhf{}
\lhead{\small\sffamily \team}
\rhead{\small\sffamily Page \thepage\ of \pageref{LastPage}}
\setlength\parskip{.5\baselineskip}

\usepackage{hyperref}
\usepackage{mathptmx}% newtxtext
\usepackage{lipsum}
\title{The \LaTeX{} Template for MCM Version 1}
\author{\small \href{http://www.latexstudio.net/}
	{\includegraphics[width=7cm]{mcmthesis-logo}}}
\date{\today}
%=========设置主要、次要文件===============
%======== The new package =============
\usepackage{subfiles}
\usepackage{enumitem}
\usepackage{tabularray}
\usepackage{subfigure}
% set for paragraph
\usepackage{indentfirst}
% use for figure
% \usepackage[section]{placeins}
% use for table
\usepackage[para]{threeparttable}
% use for paragraph
\usepackage[explicit, compact]{titlesec}
\titleformat{\paragraph}[hang]{\normalfont\normalsize\bfseries}{\theparagraph}{1em}{#1}
\titlespacing*{\paragraph}{0pt}{1.25ex plus 1ex minus .2ex}{0.1em}
\titlespacing*{\section}{0pt}{3.5ex plus 1ex minus .2ex}{1.3ex plus .2ex}
% use subparagraph for step
\titleformat{\subparagraph}[runin]{\normalfont\normalsize\bfseries}{\thesubparagraph}{1em}{\bfseries Step~#1:}
\titlespacing*{\subparagraph}{\parindent}{0pt}{0.5em}
% use for caption
\usepackage{caption} \captionsetup[figure]{justification=centering}
\captionsetup[table]{justification=centering}
% for page
\usepackage{geometry}
\geometry{left=5em,right=5em,top=5em,bottom=5em}
% for toc
\usepackage[subfigure]{tocloft}
% for center contents
\renewcommand{\cfttoctitlefont}{
  \hfill\Large\bfseries
}
\renewcommand{\cftaftertoctitle}{
  \hfill
}
\renewcommand{\cftaftertoctitleskip}{3em}
\setlength{\cftbeforesecskip}{0.2em}
\setlength{\cftbeforesubsecskip}{0.1em}
\usepackage[color=red]{attachfile2}
\setminted[python]{breaklines=true}
\begin{document}
	%% 摘要环境
	\subfile{./segments/abstract_}
	\newpage
	\tableofcontents
    \newpage
	
	
	%==========设置正文格式===================
	
	\subfile{./segments/introduction_}
	
	% 分析
	\subfile{./segments/analysis_}
	
	% 计算
	\subfile{./segments/calculate_}
	
	% 模型结果
	\subfile{./segments/model_result}
	
	% %
	% \subfile{./segments/validate_model}
	
	% % 总结
	% \subfile{./segments/conclusion_}
	
	
	% \section{A Summary}
	% \lipsum[6]
	
	% % 模型评估
	% \subfile{./segments/evaluate_mode}
	
	
	% % 模型的优点和缺点
	% \subfile{./segments/strengths_weaknesses}
    \newpage
    \addtocounter{section}{1}
	\begin{thebibliography}{99}\addcontentsline{toc}{section}{\bfseries \thesection\quad Reference}
		\bibitem{1} Zou Xiaohui, Sun Jing. LDA Topic Model[J]. Intelligent Computer and Application, 2014(5). DOI:10.3969/j.issn.2095-2163.2014.05.031.
		\bibitem{2} Tong Z, Zhang H. A text mining research based on LDA topic modelling[C]//International Conference on Computer Science, Engineering and Information Technology. 2016: 201-210.
		\bibitem{3}  Zhou Lian. The working principle and application of Word2vec[J]. Science and Technology Information Development and Economy, 2015(2):145-148. DOI:10.3969/j.issn.1005-6033.2015.02.061.
        \bibitem{4} Yang Junchuang, Zhao Chao. Review of K-Means Clustering Algorithm Research [J]. Computer Engineering and Applications, 2019,55(23):7-14,63. DOI:10.3778/j.issn.1002-8331.1908-0347 .
        \bibitem{5} Wei Jie, Li Quanming, Chu Yanyu, et al. Optimization of room-and-pillar stope layout scheme based on EWM-TOPSIS model [J]. Journal of Hefei University of Technology (Natural Science Edition), 2021,44(5):691-695. DOI :10.3969/j.issn.1003-5060.2021.05.020.
        \bibitem{6} Ji Hua. Support Vector Machine (SVM) Learning Method Based on Statistical Learning Theory[J]. Science Times, 2006(11):33-37.
        \bibitem{7} Peng Yue. Introduction of ARIMA Model[J]. Electronic World, 2014(10):259-259. DOI:10.3969/j.issn.1003-0522.2014.10.252.
        \bibitem{8} Liu Dongyang, Liu En. Improvement of Apriori Algorithm[J]. Science Technology and Engineering, 2010,10(16):4028-4031. DOI:10.3969/j.issn.1671-1815.2010.16.054.
	\end{thebibliography}
	\newpage
	\begin{appendices}
    Here are simulation programmes we used in our model as follow. Just show the main code, other code in the file shown below.
    \begin{enumerate}[topsep=0pt]
        \item Data Visualization \textattachfile{./code/Mercer-Data-visualization.py}{Mercer-Data-visualization.py}
        \item K-Means Clustering \textattachfile{./code/Mercer-Kmeans-clustering.py}{Mercer-Kmeans-clustering.py}
        \item Data Processing \textattachfile{./code/Mercer-Data-processing.py}{Mercer-Data-processing.py}
        \item LDA \textattachfile{./code/Mercer-LDA.py}{Mercer-LDA.py}
    \end{enumerate}
		\section{K-Means Clustering Algorithm}
        \begin{minted}{python}
  from numpy import *
  import matplotlib.pyplot as plt

  def loadDataSet(fileName):  
      dataMat = []              
      fr = open(fileName)
      for line in fr.readlines():
          curLine = line.strip().split('\t')
          fltLine = map(float, curLine) 
          dataMat.append(fltLine)
      return dataMat

  def distEclud(vecA, vecB):
      return sqrt(sum(power(vecA - vecB, 2)))

  def randCent(dataSet, k):
      n = shape(dataSet)[1]
      centroids = mat(zeros((k,n)))   
      for j in range(n):
          minJ = min(dataSet[:,j])
          maxJ = max(dataSet[:,j])
          rangeJ = float(maxJ - minJ)
          centroids[:,j] = minJ + rangeJ * random.rand(k, 1)
      return centroids

  def kMeans(dataSet, k, distMeans =distEclud, createCent = randCent):
      m = shape(dataSet)[0]  
      clusterAssment = mat(zeros((m,2)))
      centroids = createCent(dataSet, k)
      clusterChanged = True   
      while clusterChanged:
          clusterChanged = False;
          for i in range(m): 
              minDist = inf; minIndex = -1;
              for j in range(k):
                  distJI = distMeans(centroids[j,:], dataSet[i,:])
                  if distJI < minDist:
                      minDist = distJI; minIndex = j 
              if clusterAssment[i,0] != minIndex:
                  clusterChanged = True  
              clusterAssment[i,:] = minIndex,minDist**2   
          for cent in range(k):   
              ptsInClust = dataSet[nonzero(clusterAssment[:,0].A == cent)[0]]   
              centroids[cent,:] = mean(ptsInClust, axis = 0)  
  return centroids, clusterAssment
          \end{minted}
          \section{Data Preprocessing}
          \par
          \begin{minted}{python}
  import seaborn as sns
  sns.axes_style("darkgrid")
  explode =[0,0,0,0.3,0]
  labels =['one stars','two stars','three stars','four stars','five stars']
  colors = ['salmon','tan','darkorange','skyblue','khaki']
  patches,l_text,p_text=plt.pie(x=pacifier_stars_ratio,labels=labels,
          explode=explode,colors=colors, autopct='%.3f%%',pctdistance=0.4, 
          labeldistance=0.7,startangle=180,center=(4,4),radius=3.8,counterclock= False,

  for t in p_text:
      t.set_size(17)

  for t in l_text:
      t.set_size(17)
  plt.xticks(())

  plt.yticks(())
  plt.title('pacifier_stars_ratio',y=-1.18,fontsize=18)
  plt.legend(patches,
             labels,
             fontsize=18,
             loc="center left",
             bbox_to_anchor=(2, 0, 1, -1))

  plt.show()


  from wordcloud import WordCloud
  import matplotlib.pyplot as plt
  from PIL import Image
  wordcloud = WordCloud(mask=mask,background_color='#FFFFFF',scale=1,).generate(a)
  image_produce = wordcloud.to_image()
  wordcloud.to_file("new_wordcloud.jpg")
  image_produce.show()
  \end{minted}
  \section{LDA}
  \begin{minted}{python}
  import pandas as pd
  import numpy as np
  import matplotlib.pyplot as plt

  hair_dryer_data = pd.read_excel('hair_dryer.xlsx')
  microwave_data = pd.read_excel('microwave.xlsx')
  pacifier_data = pd.read_excel('pacifier.xlsx')

  import re


  #
  def clean_text(text):
      text = text.replace("<br />", " ")
      text = text.replace("<br", " ")
      text = re.sub(r'[^\x00-\x7F]+', ' ', text)
      text = re.sub(r"([.,!:?()])", r" \1 ", text)
      text = re.sub(r"\s{2,}", " ", text)
      text = text.replace("-", " ")
      return text


  hair_dryer_data['review_body'].apply(clean_text)

  import nltk
  from sklearn.feature_extraction.text import TfidfVectorizer, CountVectorizer
  from sklearn.decomposition import LatentDirichletAllocation

  n_features = 1000
  tf_vectorizer = CountVectorizer(strip_accents='unicode',
                                  max_features=n_features,
                                  stop_words='english',
                                  max_df=0.5,
                                  min_df=10)
  tf = tf_vectorizer.fit_transform(microwave_data['review_headline'])

  lda = LatentDirichletAllocation(n_components=15,
                                  max_iter=150,
                                  learning_method='online',
                                  learning_offset=50, random_state=0)
  lda.fit(tf)
  \end{minted}
        % \lstinputlisting[language=Python]{./code/Mercer-Kmeans-clustering2.py}
        % \lstinputlisting[language=Python]{./code/Mercer-LDA2.py}
        % \lstinputlisting[language=Python]{./code/Mercer-Data-visualization2.py}
        % \lstinputlisting[language=Python]{./code/Mercer-Data-processing2.py}
	\end{appendices}
	
	
	
	
\end{document}

%%% Local Variables:
%%% mode: latex
%%% TeX-master: t
%%% End:
