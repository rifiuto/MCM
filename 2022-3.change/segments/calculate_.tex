\documentclass[../mcmpaper]{subfiles}
\begin{document}
\section{Problem Analysis}
\subsection{Analysis of Question 1}
In order to analyze the travel characteristics of passengers in the city, we analyzed the data at different levels from seven aspects. Firstly, according to the information given by the data,such as travel time, passenger ID and payment method are classified and combined for data processing. Analyze the cumulative sum of payment times in a day, a week and different months to indicate which payment method people will choose at different times, classify the people who travel, and analyze the tendency of different people for payment methods. Under the same conditions, screen out the abnormal data that is not within the scope of our discussion, use pandas to read the data, visualize the results, get the data table, and then analyze the data to figure out the possible causes and travel characteristics.
\subsection{Analysis of Question 2}
In order to obtain the third-party profit model, it is necessary to clarify the items that the third-party payment platform needs to pay in revenue and expenditure. The revenue items are divided into advertising fees, service fees, handling fees and interest income of precipitated funds. The expenditure items are divided into: early-stage advertising expenses, mobile terminal access expenses and fixed expenses (infrastructure investment, fixed cost of new projects, employee salary, etc.). Because the profit model involves many variables and there are some other quantities that do not change with the variables, after listing the equation, we refine the variables by using determinant, power function and other functions, and finally find the data and set some quantitative values to obtain the final quantitative analysis results, Calculate the profit model of the third-party payment platform of public transport, which can well match the problem.
\subsection{Analysis of Question 3}
Since only a quarter of the data obtained after the installation of mobile payment devices in buses and subways are given in the problem, in order to establish a model that suit all situations, we need to push the installation ratio from a quarter to all. According to the result of the first question, we put forward the assumption that these quarter of the lines account for the vast majority of passenger flow and demonstrate that the assumption is tenable. Taking this result into the supply demand model to calculate the relationship between customer demand and the full promotion of third-party payment. Then use the innovation diffusion model proposed by the famous economist Rogers to spread the relationship, speculate the passenger flow when the third-party mobile payment platform is 100\% applied, and then bring it into the profit model of the second question to calculate the profit of the third-party payment platform. 
\subsection{Analysis of question 4}
On the basis of the first three questions, we make predictions and suggestions for the promotion benefits of the third-party mobile payment platform according to the results obtained from the profit model. Based on the results obtained from the results and data analysis, the feasibility opinion report on the company's development is made as an aid to the company's decision-making.which can give a feasible scheme in the most profitable situation where possible.
\end{document}
%%% Local Variables:
%%% mode: latex
%%% TeX-master: "../mcmpaper"
%%% End:
